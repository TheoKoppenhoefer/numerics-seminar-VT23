


\documentclass{report}
% alternatives: scrartcl, article or report


%%%%% PACKAGES



%%%%% PACKAGES

% small tweaks and nicer typography
\usepackage{microtype}
\usepackage{hyperref}

% changes language to German
% gives proper date, and correct hyphenation
%\usepackage[ngerman]{babel}
%\uselanguage{German}
%\languagepath{German}

% basic math stuff
\usepackage{mathtools}
\usepackage{amssymb}
\usepackage{amsthm}
%\usepackage{tikz-cd}
\usepackage{cancel}
\usepackage{cases}
\usepackage{dsfont}
\usepackage{delimset} % for nice delimiters
\usepackage{centernot}


% tikz
\usepackage{tikz}
\usepackage{pgfplots}
\usetikzlibrary{positioning}
%\usetikzlibrary{patterns}
%\usetikzlibrary{babel}
\tikzset{>=stealth}
\usepackage{wrapfig}

% Plotting
\usepackage{pgfplots}

% code
%\usepackage{listings}
%\usepackage{pythonhighlight}
%\usepackage{algorithm}
%\usepackage{algpseudocode}
\usepackage{algorithm2e}
\RestyleAlgo{algoruled}

\usepackage[backend=bibtex, style=ieee]{biblatex}
%\usepackage{biblatex}
%\addbibresource{mybib.bib}

% dealing with figures
%\usepackage[figurename=Abb.]{caption}
\usepackage{subcaption}
\usepackage{wrapfig}

% display quotes correctly
\usepackage{csquotes}

% allow for any font-size, alternative mathpazo
\usepackage{mathptmx}

% color
\usepackage{xcolor}

%%%% Graphics %%%%%

%\graphicspath{{Plots/}}

%\newcommand{\tikzmark}[3][]{\tikz[remember picture,baseline] \node [anchor=base,#1](#2) {$#3$};}

%\usepackage{booktabs}
%\usepackage{bm}
%\usepackage{minted}

% for inkscape images
%\usepackage{pdftricks}
%\begin{psinputs}
%   \usepackage{pstricks}
%   \usepackage{multido}
%\end{psinputs}
%\usepackage[pdf]{pstricks}
%\usepackage{import}



% images
\usepackage{graphicx}
\graphicspath{ {./Plots} }

% tikz
\usepackage{tikz}
\usetikzlibrary{positioning}
\usetikzlibrary{babel}
\tikzset{>=stealth}
\newcommand{\tikzmark}[3][]{\tikz[remember picture,baseline] \node [anchor=base,#1](#2) {$#3$};}


% Tikz librarys
\usetikzlibrary{datavisualization}
\usetikzlibrary{datavisualization.formats.functions}
%\usetikzlibrary{external}
%\tikzexternalize[prefix=../Resources/]



%%%%% CONFIGURATION

% prevents automatic line breaks inside of equations
% since it looks bad
\binoppenalty = \maxdimen
\relpenalty   = \maxdimen


%%%%%% PGFPLOTS %%%%%%%%%%%

%\usepgfplotslibrary{grouplots}
\usepgfplotslibrary{dateplot}


%%%%% CUSTOM COMMANDS

% real numbers via \R
% complex numbers via \C
% general field via \K
\def\C{\mathbb{C}}
\def\R{\mathbb{R}}
\def\K{\mathbb{K}}
\def\F{\mathbb{F}}
\def\Q{\mathbb{Q}}
\def\Z{\mathbb{Z}}
\def\N{\mathbb{N}}
\def\H{\mathbb{H}}
\def\e{\varepsilon}

\newcommand{\cA}{\mathcal{A}}
\newcommand{\cB}{\mathcal{B}}
\newcommand{\cC}{\mathcal{C}}
\newcommand{\cD}{\mathcal{D}}
\newcommand{\cE}{\mathcal{E}}
\newcommand{\cF}{\mathcal{F}}
\newcommand{\cG}{\mathcal{G}}
\newcommand{\cH}{\mathcal{H}}
\newcommand{\cI}{\mathcal{I}}
\newcommand{\cJ}{\mathcal{J}}
\newcommand{\cK}{\mathcal{K}}
\newcommand{\cL}{\mathcal{L}}
\newcommand{\cM}{\mathcal{M}}
\newcommand{\cN}{\mathcal{N}}
\newcommand{\cO}{\mathcal{O}}
\newcommand{\cP}{\mathcal{P}}
\newcommand{\cQ}{\mathcal{Q}}
\newcommand{\cR}{\mathcal{R}}
\newcommand{\cS}{\mathcal{S}}
\newcommand{\cT}{\mathcal{T}}
\newcommand{\cU}{\mathcal{U}}
\newcommand{\cV}{\mathcal{V}}
\newcommand{\cW}{\mathcal{W}}
\newcommand{\cX}{\mathcal{X}}
\newcommand{\cY}{\mathcal{Y}}
\newcommand{\cZ}{\mathcal{Z}}

\newcommand{\bA}{\mathbb{A}}
\newcommand{\bB}{\mathbb{B}}
\newcommand{\bC}{\mathbb{C}}
\newcommand{\bD}{\mathbb{D}}
\newcommand{\bE}{\mathbb{E}}
\newcommand{\bF}{\mathbb{F}}
\newcommand{\bG}{\mathbb{G}}
\newcommand{\bH}{\mathbb{H}}
\newcommand{\bI}{\mathbb{I}}
\newcommand{\bJ}{\mathbb{J}}
\newcommand{\bK}{\mathbb{K}}
\newcommand{\bL}{\mathbb{L}}
\newcommand{\bM}{\mathbb{M}}
\newcommand{\bN}{\mathbb{N}}
\newcommand{\bO}{\mathbb{O}}
\newcommand{\bP}{\mathbb{P}}
\newcommand{\bQ}{\mathbb{Q}}
\newcommand{\bR}{\mathbb{R}}
\newcommand{\bS}{\mathbb{S}}
\newcommand{\bT}{\mathbb{T}}
\newcommand{\bU}{\mathbb{U}}
\newcommand{\bV}{\mathbb{V}}
\newcommand{\bW}{\mathbb{W}}
\newcommand{\bX}{\mathbb{X}}
\newcommand{\bY}{\mathbb{Y}}
\newcommand{\bZ}{\mathbb{Z}}



\newcommand{\hu}{\hat{u}}
\newcommand{\hv}{\hat{v}}
\newcommand{\hV}{\hat{V}}
\newcommand{\hw}{\hat{w}}
\newcommand{\hW}{\hat{W}}
\newcommand{\hA}{\hat{A}}
\newcommand{\hC}{\hat{C}}
\newcommand{\hR}{\hat{R}}
\newcommand{\hQ}{\hat{Q}}
\newcommand{\hq}{\hat{q}}
\newcommand{\hp}{\hat{p}}
\newcommand{\hl}{\hat{\ell}}
\newcommand{\hlambda}{\hat{\lambda}}
\newcommand{\ha}{\hat{a}}
\newcommand{\hb}{\hat{b}}
\newcommand{\hs}{\hat{s}}


\newcommand{\tiS}{\tilde{S}}
\newcommand{\tiu}{\tilde{u}}
\newcommand{\tih}{\tilde{h}}
\newcommand{\tix}{\tilde{x}}
\newcommand{\tiy}{\tilde{y}}
\newcommand{\tis}{\tilde{s}}
\newcommand{\tie}{\tilde{\e}}
\newcommand{\tisigma}{\tilde{\sigma}}


\newcommand{\bartheta}{\bar{\theta}}
\newcommand{\barU}{\bar{U}}



%%%%%%%%%%    Math operators    %%%%%%%%%%%%%%%%%%%%%%%%%%%


\newcommand{\dif}[1]{\,\mathrm{d} #1}
%\newcommand{\norm}[1]{\lVert #1 \rVert}
%\newcommand{\abs}[1]{\left| #1 \right|}
\newcommand{\bnorm}[1]{\left\lVert #1\right\rVert}
\newcommand{\vii}[2]{\ensuremath{\begin{bmatrix}#1 \\ #2 \end{bmatrix}}}
\newcommand{\mii}[4]{\ensuremath{\begin{bmatrix}#1&#2 \\ #3&#4 \end{bmatrix}}}
\newcommand{\mc}[1]{\mathcal{#1}}

\newcommand{\one}{\mathds{1}}
\newcommand{\bigO}{\mathcal{O}}


\DeclareMathOperator{\Image}{Image}
\DeclareMathOperator{\Vspan}{Span}
\DeclareMathOperator{\Erf}{erf}
\DeclareMathOperator{\Id}{Id}             % identity morphism
% \DeclareMathOperator{\ker}{ker}           % kernel
\DeclareMathOperator{\rg}{rg}             % image
\DeclareMathOperator{\defekt}{def}             % defect
\DeclareMathOperator{\im}{im}             % image
\DeclareMathOperator{\Hom}{Hom}           % homomorphisms
\DeclareMathOperator{\End}{End}           % endomorphisms
\DeclareMathOperator{\Span}{Span}         % linear span
\DeclareMathOperator{\grad}{\nabla}         % gradient
\DeclareMathOperator{\diam}{diam}         % gradient
\DeclareMathOperator{\Tr}{Tr}       	  % trace
\DeclareMathOperator{\diver}{Div}			% divergence
\DeclareMathOperator{\supp}{supp}			% support
\DeclareMathOperator{\dist}{dist}			% distance
\DeclareMathOperator{\inter}{int}			% interiour
\DeclareMathOperator{\epi}{epi}			% epigraph
\DeclareMathOperator{\hyp}{hyp}			% hypograph
\DeclareMathOperator{\Lip}{Lip}			% lipschitz konstant
\DeclareMathOperator{\graph}{graph}			% graph
\DeclareMathOperator{\sgn}{sgn}			% sign
\DeclareMathOperator{\BMO}{BMO}			% BMO
\DeclareMathOperator{\mean}{mean}			% BMO
%\DeclareMathOperator{\B}{B}			% BMO


% \vect{ x // y // z } for a column vector with entries x, y, z
% similarly for larger vectors
% in this code:  1 = number of arguments
%               #1 = first argument
\newcommand{\vect}[1]{\begin{bmatrix} #1 \end{bmatrix}}

% \conj{z} for complex conjugation
\newcommand{\conj}{\overline}

%counter of current constant number:    
\newcounter{constant} 
%defines a new constant, but does not typeset anything:
\newcommand{\newconstant}[1]{\refstepcounter{constant}\label{#1}} 
%typesets named constant:
\newcommand{\useconstant}[1]{c_{\ref{#1}}}

%%%%%%% GENERAL STYLE %%%%%%%%%%%%%%%%%%

\setcounter{tocdepth}{3}
\setcounter{secnumdepth}{0}


%%%%%%% COLORS %%%%%%%%%%%%%%%%%%%%%%%%


\newcommand{\black}{\color{black}}


%%%%%% TITLE PAGE
%
%\subject{Specialised Course in Integration Theory, VT23}
%\title{Assignment Chapter 3.5}
%\author{Theo Koppenhöfer}
%\date{\today}
%
%
%%%%%% The content starts here %%%%%%%%%%%%%
%
%
%\begin{document}
%
%\maketitle
%
%
%%\nocite{*}
%\printbibliography
%
%\end{document}



% theorem-like environments
\newcounter{everything}
\newtheorem{corollary}[everything]{Corollary}
\newtheorem{lemma}[everything]{Lemma}
\newtheorem{proposition}[everything]{Proposition}
\newtheorem{theorem}[everything]{Theorem}
\newtheorem*{claim}{Claim}
\newtheorem*{given}{Given}

%%%%%%% GENERAL STYLE %%%%%%%%%%%%%%%%%%

\setcounter{tocdepth}{3}
\setcounter{secnumdepth}{0}


%%%%%% TITLE PAGE
%
%\subject{Specialised Course in Integration Theory, VT23}
%\title{Assignment Chapter 3.5}
%\author{Theo Koppenhöfer}
%\date{\today}
%
%
%%%%%% The content starts here %%%%%%%%%%%%%
%
%
%\begin{document}
%
%\maketitle
%
%
%%\nocite{*}
%\printbibliography
%
%\end{document}


%%%%% TITLE PAGE

%\subject{, VT23}
\title{ Project Report for Seminar Course in Numerical Analysis, VT23 \\[1ex]
	  \large Junzi Zhang, Brendan O'Donoghue, Stephen Boyd: Globally Convergent Type-I Anderson Acceleration for Non-Smooth Fixed-Point Iterationsy}
%\subtitle{}
\author{Theo Koppenhöfer}
\date{Lund \\[1ex] \today}

\addbibresource{bibliography.bib}

%%%%% The content starts here %%%%%%%%%%%%%


\begin{document}

\maketitle


\begin{proof}
	The proof follows \cite{ZhaAA}[Theorem 6].
	We partition $\N=K_{AA}\sqcup K_{KM}$ where $K_{AA}=\brk[c]{k_0,k_1,\dots}$ denote the indices $k$ where the algorithm chose an AA-step (a) and $K_{KM}=\brk[c]{l_0,l_1,\dots}$ where the algorithm chose a KM-step (b).
	
	\begin{center}
	\SetAlgoSkip{bigskip}
	\DontPrintSemicolon
	\begin{algorithm}[H]
		\uIf{$\norm{g_k}\leq D\barU(n_{AA}+1)^{-(1+\e)}$}{
			Set $x_{k+1}=\tix_{k+1}$ and $n_{AA}= n_{AA}+1$. \brkcomment*[r]{a}
		}
		\Else{
			Set $x_{k+1}= (1-\alpha)x_k +\alpha f(x_k)$. \brkcomment*[r]{b}
		}
	\caption{The two cases for $x_{k+1}$.}
	\end{algorithm}
	\end{center}
	Let $y$ be a fixed point. We distinguish
	\begin{description}
		\item[case (1)]
		$k\in K_{AA}$ then
		\newconstant{upperHk}
		\newconstant{CDU}
		\begin{equation}
		\begin{aligned}
			\norm{x_{k+1}-y}&\leq \norm{x_k-y}+\norm{H_kg_k} \\
			&\leq \norm{x_k-y}+\useconstant{upperHk}\norm{g_k} \\
			&\leq \norm{x_k-y}+\useconstant{CDU}(k+1)^{-(1+\e)}
			\label{eq:16}
		\end{aligned}
		\end{equation}
		\item[case (2)]
		$k\in K_{KM}$ then (motivate this)
		\begin{align}
			\norm{x_{k+1}-y}^2\leq \norm{x_k-y}^2-\alpha(1-\alpha)\norm{g_k}^2\leq \norm{x_k-y}^2
			\label{eq:17}
		\end{align}
	\end{description}
	Hence in any case
	\newconstant{E}
	\begin{align*}
		\norm{x_k-y}
		&\leq \norm{x_0-y}+\sum_{l=0}^{k-1}\norm{x_{l+1}-x_l} \\
		&\leq \norm{x_0-y}+\useconstant{CDU}\sum_k(k+1)^{-(1+\e)}
		= \useconstant{E}<\infty\,.
	\end{align*}
	It then follows that
	\begin{equation}
	\begin{aligned}
		a_{k+1} &= \norm{x_{k+1}-y}^2 \\
		&\stackrel{\eqref{eq:16}, \eqref{eq:17}}{\leq} \underbrace{\norm{x_k-y}^2}_{=a_k}+\underbrace{\useconstant{CDU}^2(k+1)^{-2(1+\e)}+2\useconstant{CDU}\underbrace{\norm{x_k-y}}_{\leq \useconstant{E}}(k+1)^{-(1+\e)}}_{=b_k} \\
		&= a_k+b_k
	\end{aligned}
	\label{eq:19}
	\end{equation}
	and hence
	\begin{align*}
		\alpha(1-\alpha)\sum_i\norm{g_{l_i}}^2
		\stackrel{\eqref{eq:17}}{\leq} \sum_i a_{l_i}-a_{l_i+1}
		\stackrel{\eqref{eq:19}}{\leq}a_0+\sum_k b_k
		<\infty
	\end{align*}
	We therefore have $\lim_i\norm{g_{l_i}}=0$. It also follows from $\norm{g_{k_i}}\leq D\barU (i+1)^{-(1+\e)}$ that $\lim_i\norm{g_{k_i}}=0$. Thus indeed $\lim_k\norm{g_k}=0$.
	
	(part 2)
	Let now $n_j$ and  $N_j\geq n_j$ be such that
	\begin{align*}
		a_{n_j}\xrightarrow{j\to\infty}\liminf_ka_k=\underline{a} \\
		a_{N_j}\xrightarrow{j\to\infty}\limsup_ka_k=\overline{a}	
	\end{align*}
	Then it follows that
	\begin{align*}
		\overline{a}-\underline{a}
		\xleftarrow{n_j\to\infty}\overline{a}-a_{n_j}
		\xleftarrow{N_j\to\infty}a_{N_j}-a_{n_j}
		= \sum_{k=n_j}^{N_j-1}a_{k+1}-a_k
		\leq \sum_{k=n_j}^\infty b_k
		\xrightarrow{n_j\to\infty}0
	\end{align*}
	so
	\begin{align*}
		\limsup_ka_k=\overline{a}\leq \underline{a}=\liminf_ka_k
	\end{align*}
	and thus $a_k=\norm{x_k-y}$ converges to some $b$.
	
	(part 3)
	Let $k_j$ and $l_j$ be convergent subsequences of $x_k$ convergent against $y_1$ and $y_2$ respectively. Since by continuity of $g$
	\begin{align*}
		\norm{g(y_1)}=\lim_j\norm{g(x_{k_j})}=0
	\end{align*}
	we have that $y_1$ is a fixed point and $y_2$ too.
	Now
	\begin{align*}
		\norm{y_1} 
		\xleftarrow{j\to\infty} \norm{x_{k_j}}^2
		= \norm{x_k-y}^2+\norm{y}^2+2y^\top x_{k_j}
		\xrightarrow{j\to\infty} b^2+\norm{y}^2+2y^\top y_1
	\end{align*}
	and analogously for $y_2$. Thus
	\begin{align*}
		\norm{y_i}=b^2+\norm{y}^2+2y^\top y_i
	\end{align*}
	which implies
	\begin{align*}
		2y^\top(y_1-y_2) = \norm{y_1}^2-\norm{y_2}^2
	\end{align*}
	It then follows from $y\in\brk[c]{y_i}_i$ that
	\begin{align*}
		y_1^\top(y_1-y_2) = y_2^\top(y_1-y_2)
	\end{align*}
	and further
	\begin{align*}
		(y_1-y_2)^\top(y_1-y_2) = 0
	\end{align*}
	and thus $y_1=y_2$. We have shown that two convergent subsequences have the same limit and hence $x_k$ is convergent and the solution must be a fixed point of $f$.
\end{proof}


\newpage
\section*{Sources}
\nocite{*}
%	\bibliographystyle{plain}
%	\bibliography{bibliography}
\printbibliography
\end{document}
